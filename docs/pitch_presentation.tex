\documentclass{beamer}
\mode<presentation> {
\usetheme{Madrid}
\setbeamertemplate{navigation symbols}{}
}
\usepackage{graphicx}
\usepackage{booktabs}
\usepackage[utf8]{inputenc}
\usepackage[T1]{fontenc}

\title[Criptografía Ligera - Pitch]{Análisis Comparativo de Algoritmos de Criptografía Ligera}
\author{Kevin Arturo Amaya Osorio}
\institute[UNAL]{
Universidad Nacional de Colombia \\
\medskip
\textit{kaamayao@unal.edu.co}
}
\date{\today}

\begin{document}

\begin{frame}
\titlepage
\end{frame}

\begin{frame}
\frametitle{¿Qué es la Criptografía Ligera?}

\begin{block}{Definición}
La \textbf{criptografía ligera} (Lightweight Cryptography) es una rama de la criptografía que diseña algoritmos optimizados para dispositivos con recursos limitados.
\end{block}

\textbf{Características principales:}
\begin{itemize}
\item Bajo consumo de memoria (RAM y ROM)
\item Menor consumo energético
\item Implementaciones compactas en hardware
\item Adecuada para IoT, RFID, sensores y sistemas embebidos
\end{itemize}

\end{frame}

\begin{frame}
\frametitle{Cifrado Simétrico y Cifrado de Bloque}

\begin{block}{Cifrado Simétrico}
Un esquema de cifrado simétrico es una tupla $(\mathcal{K}, \mathcal{M}, \mathcal{C}, E, D)$ donde:
\begin{itemize}
\item $\mathcal{K}$ es el espacio de claves, $\mathcal{M}$ el espacio de mensajes, $\mathcal{C}$ el espacio de textos cifrados
\item $E: \mathcal{K} \times \mathcal{M} \rightarrow \mathcal{C}$ es la función de cifrado
\item $D: \mathcal{K} \times \mathcal{C} \rightarrow \mathcal{M}$ es la función de descifrado
\item $\forall k \in \mathcal{K}, \forall m \in \mathcal{M}: D(k, E(k, m)) = m$
\end{itemize}
\end{block}

\vspace{-0.3cm}

\begin{block}{Cifrado de Bloque}
Es un cifrado simétrico donde $\mathcal{M} = \mathcal{C} = \{0,1\}^n$ para un tamaño de bloque $n$ fijo.
\begin{itemize}
\item $E_k: \{0,1\}^n \rightarrow \{0,1\}^n$ es una permutación (biyección)
\item Para cada clave $k$, existe $E_k^{-1} = D_k$
\item Tamaños típicos: $n = 64$ (DES, PRESENT) o $n = 128$ (AES)
\end{itemize}
\end{block}

\end{frame}


\begin{frame}
\frametitle{Estructura: Red de Feistel}

\begin{columns}[T]
\begin{column}{0.4\textwidth}
\includegraphics[height=0.7\textheight]{images/feistel_cipher.pdf}
\end{column}
\begin{column}{0.6\textwidth}
\textbf{Funcionamiento:}
\begin{itemize}
\item Divide el bloque en dos mitades (L, R)
\item Aplica función F con subclave en cada ronda
\item Resultado se XOR con la otra mitad
\item Las mitades se intercambian
\end{itemize}

\vspace{0.3cm}

\textbf{Ventaja clave:}
\begin{itemize}
\item El mismo circuito sirve para cifrar y descifrar
\end{itemize}

\vspace{0.3cm}

\textbf{Usado en:} DES, 3DES, Blowfish, Twofish

\vspace{0.3cm}

\begin{block}{Seguridad}
Teorema de Luby-Rackoff (1988)
\end{block}
\end{column}
\end{columns}

\end{frame}

\begin{frame}
\frametitle{Estructura: Red de Sustitución-Permutación (SPN)}

\begin{columns}[T]
\begin{column}{0.4\textwidth}
\includegraphics[height=0.65\textheight]{images/spn_network.png}
\end{column}
\begin{column}{0.6\textwidth}
\textbf{Componentes:}
\begin{itemize}
\item \textbf{S-box}: Sustitución no lineal (confusión)
\item \textbf{P-box}: Permutación de bits (difusión)
\item \textbf{Key mixing}: XOR con subclave de ronda
\end{itemize}

\vspace{0.3cm}

\textbf{Ventaja clave:}
\begin{itemize}
\item Alta paralelización en hardware
\item Mayor velocidad en implementaciones optimizadas
\end{itemize}

\vspace{0.3cm}

\textbf{Usado en:} AES, PRESENT, SHARK

\vspace{0.3cm}

\begin{block}{Seguridad}
Wide Trail Strategy (Daemen-Rijmen, 2001)\end{block}
\end{column}
\end{columns}

\end{frame}

\begin{frame}
\frametitle{Los Tres Algoritmos Evaluados}

\begin{columns}[T]
\begin{column}{0.33\textwidth}
\begin{block}{\textbf{DES} (1977)}
\scriptsize
\begin{itemize}
\item IBM + NSA
\item Red de Feistel
\item Clave: 56 bits
\item Bloque: 64 bits
\item 16 rondas
\item \textcolor{red}{Roto en 22h (1999)}
\end{itemize}
\end{block}
\end{column}

\begin{column}{0.33\textwidth}
\begin{block}{\textbf{AES-128} (2001)}
\scriptsize
\begin{itemize}
\item Competencia NIST
\item Red SPN
\item Clave: 128 bits
\item Bloque: 128 bits
\item 10 rondas
\item \textcolor{green}{Estándar actual}
\end{itemize}
\end{block}
\end{column}

\begin{column}{0.33\textwidth}
\begin{block}{\textbf{PRESENT} (2007)}
\scriptsize
\begin{itemize}
\item Orange Labs + Uni
\item Red SPN
\item Clave: 80 bits
\item Bloque: 64 bits
\item 31 rondas
\item \textcolor{blue}{1570 GE (ultra-ligero)}
\end{itemize}
\end{block}
\end{column}
\end{columns}

\vspace{0.3cm}

\textbf{Aplicaciones:}
\begin{itemize}
\item \textbf{DES}: Cajeros ATM legacy, tarjetas EMV antiguas (deprecado)
\item \textbf{AES}: WiFi, HTTPS, WhatsApp, BitLocker, aceleración AES-NI
\item \textbf{PRESENT}: ISO/IEC 29192-2, influenció ASCON (NIST 2023)
\end{itemize}

\end{frame}

\begin{frame}
\frametitle{Resultados: Rendimiento y Latencia}

\begin{columns}[T]
\begin{column}{0.5\textwidth}
\includegraphics[width=\textwidth]{../results/throughput_benchmark.png}
\end{column}
\begin{column}{0.5\textwidth}
\includegraphics[width=\textwidth]{../results/latency_benchmark.png}
\end{column}
\end{columns}

\vspace{0.3cm}

\textbf{Observaciones:}
\begin{itemize}
\item \textbf{AES-128}: 68 KB/s cifrado — El más rápido en software
\item \textbf{DES}: 63 KB/s — Balanceado cifrado/descifrado
\item \textbf{PRESENT}: 38 KB/s — Diseñado para hardware, no software
\item En hardware dedicado, PRESENT sería más competitivo
\end{itemize}

\end{frame}

\begin{frame}
\frametitle{Resultados: Memoria RAM}

\begin{columns}[T]
\begin{column}{0.55\textwidth}
\includegraphics[width=\textwidth]{../results/ram_benchmark.png}
\end{column}
\begin{column}{0.45\textwidth}
\textbf{Memoria pico durante cifrado:}
\begin{itemize}
\item \textbf{AES-128}: 17 KB
\item \textbf{PRESENT}: 74 KB
\item \textbf{DES}: 80 KB
\end{itemize}

\vspace{0.3cm}

\textbf{¿Por qué importa?}
\begin{itemize}
\item MSP430: 512 bytes RAM
\item ATtiny85: 512 bytes RAM
\item ESP8266: 80 KB RAM
\end{itemize}

\vspace{0.3cm}

\begin{alertblock}{Nota}
Implementación Python. En C/ASM los valores serían menores.
\end{alertblock}
\end{column}
\end{columns}

\end{frame}

\begin{frame}
\frametitle{Resultados: Consumo Energético}

\begin{columns}[T]
\begin{column}{0.55\textwidth}
\includegraphics[width=\textwidth]{../results/power_benchmark.png}
\end{column}
\begin{column}{0.45\textwidth}
\textbf{Métricas:}
\begin{itemize}
\item Potencia (mW): similar entre los tres
\item Energía por byte (µJ/byte): varía significativamente
\end{itemize}

\vspace{0.3cm}

\textbf{Implicaciones para IoT:}
\begin{itemize}
\item Menor energía/byte = Mayor duración de batería
\item Sensores remotos: años sin recargar
\item PRESENT optimizado para eficiencia en hardware
\end{itemize}
\end{column}
\end{columns}

\end{frame}

\begin{frame}
\frametitle{Seguridad: Ataques Conocidos}

\begin{table}[h]
\centering
\begin{tabular}{lccc}
\toprule
\textbf{Algoritmo} & \textbf{Mejor Ataque} & \textbf{Complejidad} & \textbf{¿Práctico?} \\
\midrule
DES & Fuerza bruta & $2^{56}$ & \textcolor{red}{Sí (horas)} \\
AES-128 & Biclique & $2^{126.1}$ & \textcolor{green}{No} \\
PRESENT-80 & Fuerza bruta & $2^{80}$ & \textcolor{blue}{No (aún)} \\
\bottomrule
\end{tabular}
\end{table}

\vspace{0.3cm}

\textbf{Contexto:}
\begin{itemize}
\item \textbf{DES}: EFF lo rompió en 1998 con hardware de \$250,000 USD
\item \textbf{AES-128}: Biclique reduce de $2^{128}$ a $2^{126.1}$ — aún impracticable
\item \textbf{PRESENT-80}: 80 bits es adecuado para IoT de corta vida, pero no para datos de largo plazo
\end{itemize}

\end{frame}

\begin{frame}
\frametitle{Conclusiones: ¿Cuál Elegir?}

\begin{table}[h]
\centering
\scriptsize
\begin{tabular}{lcccc}
\toprule
\textbf{Algoritmo} & \textbf{Throughput} & \textbf{RAM} & \textbf{Seguridad} & \textbf{Recomendación} \\
\midrule
DES & 63 KB/s & 80 KB & Obsoleto & \textcolor{red}{No usar} \\
AES-128 & 68 KB/s & 17 KB & Excelente & \textcolor{green}{Uso general} \\
PRESENT-80 & 38 KB/s & 74 KB & Adecuado & \textcolor{blue}{IoT restringido} \\
\bottomrule
\end{tabular}
\end{table}

\vspace{0.3cm}

\begin{block}{Recomendaciones}
\begin{itemize}
\item \textbf{DES}: Solo valor histórico/educativo. Deprecado desde 2005.
\item \textbf{AES-128}: Opción por defecto. Si tienes AES-NI, úsalo.
\item \textbf{PRESENT-80}: Para RFID, sensores desechables, smart dust.
\end{itemize}
\end{block}

\vspace{0.3cm}

\begin{alertblock}{El Futuro: ASCON}
NIST estandarizó \textbf{ASCON} en 2023 para criptografía ligera autenticada. Hereda ideas de PRESENT.
\end{alertblock}

\end{frame}

\begin{frame}
\frametitle{Mensaje Final}

\begin{center}
\Large
\textbf{La criptografía ligera no es simplemente hacer algoritmos más pequeños.}

\vspace{0.8cm}

\normalsize
Es el arte de encontrar el \textbf{balance óptimo} entre tres factores en tensión:

\vspace{0.5cm}

\begin{columns}[c]
\begin{column}{0.3\textwidth}
\centering
\textbf{Seguridad}
\end{column}
\begin{column}{0.3\textwidth}
\centering
\textbf{Rendimiento}
\end{column}
\begin{column}{0.3\textwidth}
\centering
\textbf{Recursos}
\end{column}
\end{columns}

\vspace{0.8cm}

\small
Cada aplicación IoT tiene restricciones diferentes.\\
Elegir el algoritmo correcto requiere entender tanto las limitaciones del hardware como los requisitos de seguridad.

\vspace{1cm}

\normalsize
\end{center}

\end{frame}

\end{document}

\documentclass{beamer}
\mode<presentation> {
\usetheme{Madrid}
}
\usepackage{graphicx}
\usepackage{booktabs}
\usepackage{hyperref}
\title[Criptografía Ligera]{Análisis Comparativo de Algoritmos de Criptografía Ligera}
\author{Kevin Arturo Amaya Osorio}
\institute[UNAL]{
Universidad Nacional de Colombia \\
\medskip
\textit{kaamayao@unal.edu.co}
}
\date{\today}

\begin{document}

\begin{frame}
\titlepage
\end{frame}

\begin{frame}
\frametitle{Índice}
\tableofcontents
\end{frame}

\section{Introducción}

\begin{frame}
\frametitle{¿Qué es la Criptografía Ligera? \cite{survey2024}}

\begin{block}{Definición}
La \textbf{criptografía ligera} (Lightweight Cryptography) es una rama de la criptografía que diseña algoritmos optimizados para dispositivos con recursos limitados.
\end{block}

\textbf{Características principales:}
\begin{itemize}
\item Bajo consumo de memoria (RAM y ROM)
\item Menor consumo energético
\item Implementaciones compactas en hardware
\item Adecuada para IoT, RFID, sensores y sistemas embebidos
\end{itemize}

\end{frame}

\begin{frame}
\frametitle{Cifrado Simétrico y Cifrado de Bloque}

\begin{block}{Cifrado Simétrico}
Un esquema de cifrado simétrico es una tupla $(\mathcal{K}, \mathcal{M}, \mathcal{C}, E, D)$ donde:
\begin{itemize}
\item $\mathcal{K}$ es el espacio de claves, $\mathcal{M}$ el espacio de mensajes, $\mathcal{C}$ el espacio de textos cifrados
\item $E: \mathcal{K} \times \mathcal{M} \rightarrow \mathcal{C}$ es la función de cifrado
\item $D: \mathcal{K} \times \mathcal{C} \rightarrow \mathcal{M}$ es la función de descifrado
\item $\forall k \in \mathcal{K}, \forall m \in \mathcal{M}: D(k, E(k, m)) = m$
\end{itemize}
\end{block}

\vspace{-0.3cm}

\begin{block}{Cifrado de Bloque}
Es un cifrado simétrico donde $\mathcal{M} = \mathcal{C} = \{0,1\}^n$ para un tamaño de bloque $n$ fijo.
\begin{itemize}
\item $E_k: \{0,1\}^n \rightarrow \{0,1\}^n$ es una permutación (biyección)
\item Para cada clave $k$, existe $E_k^{-1} = D_k$
\item Tamaños típicos: $n = 64$ (DES, PRESENT) o $n = 128$ (AES)
\end{itemize}
\end{block}

\end{frame}

\begin{frame}
\frametitle{Confusión y Difusión (Shannon, 1945)}

\begin{block}{Confusión}
Hace que la relación entre el texto cifrado y la clave sea lo más compleja posible.
\begin{itemize}
\item Cada bit del texto cifrado debe depender de varios bits de la clave
\item Se logra principalmente con \textbf{S-boxes} (tablas de sustitución)
\item Objetivo: que cambiar un bit de la clave cambie muchos bits del resultado
\end{itemize}
\end{block}

\begin{block}{Difusión}
Dispersa la influencia de cada bit del texto plano a lo largo del texto cifrado.
\begin{itemize}
\item Cada bit del texto plano debe afectar muchos bits del texto cifrado
\item Se logra con \textbf{permutaciones} y \textbf{operaciones de mezcla}
\item Objetivo: que cambiar un bit del mensaje cambie ~50\% del resultado
\end{itemize}
\end{block}

\end{frame}

\section{Estructuras de Cifrado}

\begin{frame}
\frametitle{Red de Feistel \cite{des1977}}

\begin{columns}[T]
\begin{column}{0.4\textwidth}
\includegraphics[height=0.75\textheight]{images/feistel_cipher.pdf}
\end{column}
\begin{column}{0.6\textwidth}
\textbf{Características:}
\begin{itemize}
\item Divide el bloque en dos mitades (L, R)
\item Aplica función F con subclave en cada ronda
\item Resultado se XOR con la otra mitad
\item Las mitades se intercambian
\item \textbf{Ventaja}: Cifrado y descifrado usan la misma estructura
\end{itemize}

\textbf{Usado en:} DES, 3DES, Blowfish, Twofish
\end{column}
\end{columns}

\tiny{Imagen: Wikimedia Commons (CC BY-SA 3.0)}
\end{frame}

\begin{frame}
\frametitle{Red de Sustitución-Permutación (SPN) \cite{aes2001}}

\begin{columns}[T]
\begin{column}{0.4\textwidth}
\includegraphics[height=0.7\textheight]{images/spn_network.png}
\end{column}
\begin{column}{0.6\textwidth}
\textbf{Componentes:}
\begin{itemize}
\item \textbf{S-box}: Sustitución no lineal (confusión)
\item \textbf{P-box}: Permutación de bits (difusión)
\item \textbf{Key mixing}: XOR con subclave de ronda
\end{itemize}

\textbf{Características:}
\begin{itemize}
\item Múltiples rondas de S-P-Key
\item Alta difusión y confusión
\item Paralelizable en hardware
\end{itemize}

\textbf{Usado en:} AES, PRESENT, SHARK
\end{column}
\end{columns}

\tiny{Imagen: Wikimedia Commons (CC BY-SA 3.0)}
\end{frame}

\begin{frame}
\frametitle{Fundamentos Matemáticos de Seguridad}

\begin{block}{Teorema de Luby-Rackoff (1988) - Feistel \cite{lubyrackoff1988}}
Si la función de ronda $f$ es una \textbf{función pseudoaleatoria} (PRF) segura:
\begin{itemize}
\item \textbf{3 rondas} $\Rightarrow$ Permutación pseudoaleatoria (PRP)
\item \textbf{4 rondas} $\Rightarrow$ Permutación pseudoaleatoria fuerte (SPRP)
\end{itemize}
Esto garantiza seguridad demostrable para redes de Feistel.
\end{block}

\begin{block}{Estrategia Wide Trail (2001) - SPN \cite{widetrail2001}}
Diseñada por Daemen y Rijmen para AES:
\begin{itemize}
\item Maximiza el número de \textbf{S-boxes activas} en cualquier camino diferencial/lineal
\item Usa el \textbf{branch number} para medir difusión
\item Proporciona cotas demostrables contra criptoanálisis diferencial y lineal
\end{itemize}
\end{block}

\end{frame}

\begin{frame}
\frametitle{Feistel vs SPN: Comparación}

\begin{table}[h]
\centering
\scriptsize
\begin{tabular}{lcc}
\toprule
\textbf{Característica} & \textbf{Feistel} & \textbf{SPN} \\
\midrule
Cifrado = Descifrado & Sí (mismo código) & No (inversos) \\
Paralelismo & Menor (mitad del bloque) & Mayor (bloque completo) \\
S-box invertible & No requerido & Requerido \\
Uso de recursos & Menor & Mayor \\
Velocidad en hardware & Menor & Mayor \\
Seguridad demostrable & Luby-Rackoff (1988) & Wide Trail (2001) \\
\bottomrule
\end{tabular}
\end{table}

\textbf{Trade-offs clave:}
\begin{itemize}
\item \textbf{Feistel}: Ideal para implementaciones con recursos limitados donde el mismo circuito sirve para cifrar y descifrar
\item \textbf{SPN}: Ideal para alto rendimiento en hardware paralelo, pero requiere más área/código
\end{itemize}

\end{frame}

\section{Características de los Algoritmos}

\begin{frame}
\frametitle{DES - Historia \cite{des1977}}

\begin{block}{Origen}
Desarrollado por IBM (1973-1974) basado en el cifrado \textbf{Lucifer} de Horst Feistel.
\end{block}

\textbf{Cronología:}
\begin{itemize}
\item \textbf{1973}: NBS (ahora NIST) solicita propuestas para un estándar de cifrado
\item \textbf{1974}: IBM presenta una versión modificada de Lucifer
\item \textbf{1975}: NSA colabora con IBM, reduce la clave de 112 a 56 bits
\item \textbf{1977}: Publicado como FIPS PUB 46
\item \textbf{1999}: Roto por fuerza bruta en 22 horas (EFF)
\item \textbf{2002}: Reemplazado oficialmente por AES
\end{itemize}

\begin{alertblock}{Controversia}
El diseño de las S-boxes fue clasificado. En 1990 se descubrió que resistían criptoanálisis diferencial, técnica que la NSA aparentemente conocía desde 1977.
\end{alertblock}

\end{frame}

\begin{frame}
\frametitle{DES/3DES - Aplicaciones}

\begin{itemize}
\item Cifrado de PINs en cajeros ATM (estándar PCI PIN Security)
\item Tarjetas EMV con chip — estándar global desde los años 90
\item Sistemas de mensajería financiera interbancaria (SWIFT legacy)
\item Terminales de punto de venta (POS) en comercios
\item Sistemas legacy en bancos colombianos (migración a AES en curso)
\end{itemize}

\begin{alertblock}{Estado actual}
NIST deprecó 3DES en 2023. EMVCo está migrando a AES.
\end{alertblock}

\end{frame}

\begin{frame}
\frametitle{DES - Características \cite{des1977, survey2024}}

\begin{table}[h]
\centering
\begin{tabular}{ll}
\toprule
\textbf{Característica} & \textbf{Valor} \\
\midrule
Año de publicación & 1977 \\
Estructura & Red de Feistel \\
Tamaño de clave & 56 bits (64 con paridad) \\
Tamaño de bloque & 64 bits \\
Número de rondas & 16 \\
Nivel de seguridad & Bajo (obsoleto) \\
\bottomrule
\end{tabular}
\end{table}

\begin{alertblock}{Nota}
 El DES fue declarado obsoleto en 2005 y es vulnerable a ataques de fuerza bruta con hardware moderno.
\end{alertblock}

\end{frame}

\begin{frame}
\frametitle{DES - Implementación}

\begin{enumerate}
\item \textbf{Permutación inicial (IP)}: Reordena los 64 bits de entrada
\item \textbf{16 rondas Feistel}:
\begin{itemize}
\item Bloque dividido en L (32 bits) y R (32 bits)
\item $L_i = R_{i-1}$, \quad $R_i = L_{i-1} \oplus f(R_{i-1}, K_i)$
\end{itemize}
\item \textbf{Función f}: Expansión (32→48 bits) → XOR con subclave → 8 S-boxes (6→4 bits) → Permutación P
\item \textbf{Permutación final (IP$^{-1}$)}: Inversa de IP
\end{enumerate}

\begin{block}{Key Schedule}
56 bits de clave → 16 subclaves de 48 bits mediante rotaciones y permutaciones (PC-1, PC-2).
\end{block}

\end{frame}

\begin{frame}
\frametitle{AES - Historia \cite{aes2001}}

\begin{block}{La Competencia NIST (1997-2000)}
Proceso abierto de 5 años para reemplazar DES. 15 candidatos de 12 países fueron evaluados por la comunidad criptográfica mundial.
\end{block}

\textbf{Cronología:}
\begin{itemize}
\item \textbf{1997}: NIST anuncia competencia para nuevo estándar
\item \textbf{1998}: 15 candidatos presentados (Round 1)
\item \textbf{1999}: 5 finalistas seleccionados (Round 2)
\item \textbf{Oct 2000}: \textbf{Rijndael} es anunciado ganador
\item \textbf{Nov 2001}: Publicado como FIPS PUB 197
\end{itemize}

\begin{block}{Creadores}
Joan Daemen y Vincent Rijmen (Bélgica). El nombre ``Rijndael'' combina sus apellidos.
\end{block}

\end{frame}

\begin{frame}
\frametitle{AES - Aplicaciones}

\begin{itemize}
\item WiFi WPA2/WPA3 - estándar IEEE 802.11i
\item HTTPS/TLS 1.3 - navegación web segura
\item BitLocker, FileVault, VeraCrypt - cifrado de disco
\item WhatsApp, Signal - mensajería end-to-end (documentado)
\item Bancolombia: 23,000 dispositivos con BitLocker (Microsoft, 2021)
\end{itemize}

\begin{block}{Hardware}
Intel AES-NI, AMD-V, ARM Cryptography Extensions - aceleración nativa en procesadores modernos.
\end{block}

\end{frame}

\begin{frame}
\frametitle{AES-128 - Características \cite{aes2001, survey2024}}

\begin{table}[h]
\centering
\begin{tabular}{ll}
\toprule
\textbf{Característica} & \textbf{Valor} \\
\midrule
Año de publicación & 2001 \\
Estructura & Red de Sustitución-Permutación (SPN) \\
Tamaño de clave & 128 bits \\
Tamaño de bloque & 128 bits \\
Número de rondas & 10 \\
Nivel de seguridad & Excelente \\
\bottomrule
\end{tabular}
\end{table}

\begin{block}{Nota}
AES es el estándar actual para cifrado simétrico (FIPS 197). Soporta también claves de 192 y 256 bits.
\end{block}

\end{frame}

\begin{frame}
\frametitle{AES-128 - Implementación}

\begin{enumerate}
\item \textbf{AddRoundKey}: XOR estado con subclave
\item \textbf{10 rondas} (última sin MixColumns):
\begin{itemize}
\item \textbf{SubBytes}: Sustitución via S-box (inversión en $GF(2^8)$ + transformación afín)
\item \textbf{ShiftRows}: Rotación de filas (0,1,2,3 posiciones)
\item \textbf{MixColumns}: Multiplicación en $GF(2^8)$ por columna
\item \textbf{AddRoundKey}: XOR con subclave de ronda
\end{itemize}
\end{enumerate}

\begin{block}{Key Schedule}
128 bits → 11 subclaves de 128 bits usando RotWord, SubWord y Rcon.
\end{block}

\end{frame}

\begin{frame}
\frametitle{PRESENT - Historia \cite{present2007}}

\begin{block}{Origen}
Desarrollado en 2007 por Orange Labs (Francia), Ruhr University Bochum (Alemania) y Technical University of Denmark.
\end{block}

\textbf{Cronología:}
\begin{itemize}
\item \textbf{2007}: Presentado en CHES 2007 (Vienna)
\item \textbf{2012}: Estandarizado en ISO/IEC 29192-2
\item \textbf{2014}: Incluido en ISO/IEC 29167-11 para RFID
\item \textbf{2019}: Actualizado en ISO/IEC 29192-2:2019
\end{itemize}

\begin{block}{Objetivo de Diseño}
Cifrado \textbf{ultra-ligero} donde seguridad y eficiencia en hardware son igualmente importantes. Solo 1570 GE (gate equivalents), competitivo con cifrados de flujo.
\end{block}

\end{frame}

\begin{frame}
\frametitle{PRESENT - Aplicaciones y Legado}

\begin{itemize}
\item Estándar ISO/IEC 29192-2 e ISO/IEC 29167-11
\item Sin implementaciones comerciales confirmadas
\item Su S-box y diseño influenciaron: GIFT, LED, PHOTON, SPONGENT
\item \textbf{GIFT-COFB} (basado en PRESENT) fue finalista NIST 2021
\end{itemize}

\begin{alertblock}{Futuro}
NIST seleccionó \textbf{ASCON} como estándar (2023). PRESENT no se usa directamente, pero su legado vive en cifrados modernos.
\end{alertblock}

\end{frame}

\begin{frame}
\frametitle{PRESENT-80 - Características \cite{present2007, survey2024}}

\begin{table}[h]
\centering
\begin{tabular}{ll}
\toprule
\textbf{Característica} & \textbf{Valor} \\
\midrule
Año de publicación & 2007 \\
Estructura & Red de Sustitución-Permutación (SPN) \\
Tamaño de clave & 80 bits (también 128 bits) \\
Tamaño de bloque & 64 bits \\
Número de rondas & 31 \\
Nivel de seguridad & Adecuado para IoT \\
\bottomrule
\end{tabular}
\end{table}

\begin{block}{Nota}
PRESENT es un cifrado ultra-ligero diseñado para dispositivos con recursos limitados (ISO/IEC 29192-2:2012).
\end{block}

\end{frame}

\begin{frame}
\frametitle{PRESENT-80 - Implementación}

\begin{enumerate}
\item \textbf{31 rondas}:
\begin{itemize}
\item \textbf{AddRoundKey}: XOR estado (64 bits) con subclave
\item \textbf{S-box layer}: 16 S-boxes de 4 bits en paralelo
\item \textbf{pLayer}: Permutación de bits (bit $i$ → bit $P(i)$)
\end{itemize}
\item \textbf{AddRoundKey final}: XOR con subclave 32
\end{enumerate}

\begin{block}{Características de diseño}
\begin{itemize}
\item S-box de 4 bits: máxima no-linealidad, mínimo hardware
\item pLayer: solo cableado (0 GE en hardware)
\item Total: 1570 GE - comparable a cifrados de flujo
\end{itemize}
\end{block}

\end{frame}

\begin{frame}
\frametitle{Comparación de Características \cite{survey2024}}

\begin{table}[h]
\centering
\scriptsize
\begin{tabular}{lccc}
\toprule
\textbf{Característica} & \textbf{DES} & \textbf{AES-128} & \textbf{PRESENT-80} \\
\midrule
Año de publicación & 1977 & 2001 & 2007 \\
Estructura & Feistel & SPN & SPN \\
Tamaño de clave (bits) & 56 & 128 & 80 \\
Tamaño de bloque (bits) & 64 & 128 & 64 \\
Número de rondas & 16 & 10 & 31 \\
Nivel de seguridad & Bajo & Excelente & Adecuado IoT \\
\bottomrule
\end{tabular}
\caption{Características estáticas de los algoritmos evaluados}
\end{table}

\textbf{SPN} = Red de Sustitución-Permutación

\end{frame}

\section{Benchmarks de Rendimiento}

\begin{frame}
\frametitle{Uso de Memoria \cite{bernstein2005}}

\begin{columns}[T]
\begin{column}{0.55\textwidth}
\includegraphics[width=\textwidth]{../results/memory_benchmark.png}
\end{column}
\begin{column}{0.45\textwidth}
\textbf{Metodología:}
\begin{itemize}
\item \textbf{Memoria estática}: Tamaño de constantes (S-boxes, tablas de permutación) usando \texttt{sys.getsizeof()}
\item \textbf{Memoria dinámica}: Pico de memoria durante cifrado/descifrado usando \texttt{tracemalloc}
\item 100 iteraciones por algoritmo
\end{itemize}
\end{column}
\end{columns}
\end{frame}

\begin{frame}
\frametitle{Latencia \cite{sotocruz2024}}

\begin{columns}[T]
\begin{column}{0.55\textwidth}
\includegraphics[width=\textwidth]{../results/latency_benchmark.png}
\end{column}
\begin{column}{0.45\textwidth}
\textbf{Metodología:}
\begin{itemize}
\item Latencia por byte (µs/byte)
\item 5000 iteraciones por algoritmo
\item Fórmula:
\end{itemize}
\small
\[
\text{latencia} = \frac{\text{tiempo}}{n \times \text{bytes}} \times 10^6
\]
\normalsize
\textbf{Interpretación:}
\begin{itemize}
\item Menor latencia = mejor rendimiento
\item AES-128 es el más rápido (optimizado)
\item PRESENT-80 prioriza bajo consumo sobre velocidad
\end{itemize}
\end{column}
\end{columns}
\end{frame}

\begin{frame}
\frametitle{Throughput \cite{sotocruz2024}}

\begin{columns}[T]
\begin{column}{0.55\textwidth}
\includegraphics[width=\textwidth]{../results/throughput_benchmark.png}
\end{column}
\begin{column}{0.45\textwidth}
\textbf{Metodología:}
\begin{itemize}
\item Throughput en KB/s
\item 5000 iteraciones por algoritmo
\item Fórmula:
\end{itemize}
\small
\[
\text{throughput} = \frac{n \times \text{bytes}}{\text{tiempo}}
\]
\normalsize
\textbf{Interpretación:}
\begin{itemize}
\item Mide cuántos bytes se pueden cifrar/descifrar por segundo
\item AES-128: 68 KB/s cifrado, 34 KB/s descifrado
\item DES: ~63 KB/s balanceado
\item PRESENT: ~38 KB/s (diseñado para hardware, no software)
\end{itemize}
\end{column}
\end{columns}
\end{frame}

\begin{frame}
\frametitle{Consumo de Potencia y Energía \cite{sotocruz2024}}

\begin{columns}[T]
\begin{column}{0.55\textwidth}
\includegraphics[width=\textwidth]{../results/power_benchmark.png}
\end{column}
\begin{column}{0.45\textwidth}
\textbf{Metodología:}
\begin{itemize}
\item Basado en utilización de CPU
\item 5000 iteraciones por algoritmo
\item Fórmulas:
\end{itemize}
\small
\[
\text{Potencia} = V \times I
\]
\[
\text{Energía} = P \times \text{latencia}
\]


\normalsize
\textbf{Interpretación:}
\textbf{Para IoT:} Menor energía = mayor duración de batería
\end{column}
\end{columns}
\end{frame}

\begin{frame}
\frametitle{Uso de Memoria RAM \cite{sotocruz2024}}

\begin{columns}[T]
\begin{column}{0.55\textwidth}
\includegraphics[width=\textwidth]{../results/ram_benchmark.png}
\end{column}
\begin{column}{0.45\textwidth}
\textbf{Metodología:}
\begin{itemize}
\item Memoria pico durante ejecución
\item 100 iteraciones por algoritmo
\item Medición con \texttt{tracemalloc}
\end{itemize}
\normalsize
\textbf{Interpretación:}
\begin{itemize}
\item \textbf{AES-128}: Más eficiente en RAM (17 KB enc, 14 KB dec)
\item \textbf{DES}: Mayor consumo (80 KB enc, 58 KB dec)
\item \textbf{PRESENT}: Intermedio (74 KB enc, 52 KB dec)
\end{itemize}
\textbf{Para IoT:} Dispositivos como MSP430 solo tienen 512 bytes de RAM
\end{column}
\end{columns}
\end{frame}

\section{Conclusiones}

\begin{frame}
\frametitle{Ataques Conocidos Más Efectivos}

\begin{table}[h]
\centering
\begin{tabular}{lccc}
\toprule
\textbf{Algoritmo} & \textbf{Ataque más efectivo} & \textbf{Complejidad} & \textbf{¿Práctico?} \\
\midrule
DES & Fuerza bruta & $2^{56}$ & Sí (horas) \\
AES-128 & Biclique & $2^{126.1}$ & No \\
PRESENT-80 & Fuerza bruta & $2^{80}$ & No (aún) \\
\bottomrule
\end{tabular}
\end{table}

\begin{block}{Observaciones}
\begin{itemize}
\item \textbf{DES}: Roto por EFF en 1998 (22 horas con hardware dedicado)
\item \textbf{AES-128}: Biclique reduce de $2^{128}$ a $2^{126.1}$ — aún impracticable
\item \textbf{PRESENT-80}: Margen de 80 bits es adecuado para IoT, pero no para largo plazo
\end{itemize}
\end{block}

\end{frame}

\section{Bibliografía}

\begin{frame}
\frametitle{Bibliografía (1/2)}

\begin{thebibliography}{99}
\scriptsize

\bibitem{present2007}
Bogdanov, A., et al. (2007).
\textit{PRESENT: An Ultra-Lightweight Block Cipher}.
CHES 2007, LNCS 4727, pp. 450-466.

\bibitem{survey2024}
Chatziadam, C., et al. (2025).
\textit{A Survey on Efficient Lightweight Cryptography}.
Technologies, 13(1), 3. MDPI.

\bibitem{aes2001}
Daemen, J., \& Rijmen, V. (2001).
\textit{The Design of Rijndael: AES}.
Springer-Verlag.

\bibitem{widetrail2001}
Daemen, J., \& Rijmen, V. (2001).
\textit{The Wide Trail Design Strategy}.
LNCS 2260, pp. 222-238.

\bibitem{lubyrackoff1988}
Luby, M., \& Rackoff, C. (1988).
\textit{How to Construct Pseudorandom Permutations from PRFs}.
SIAM J. Computing, 17(2), pp. 373-386.

\end{thebibliography}

\end{frame}

\begin{frame}
\frametitle{Bibliografía (2/2)}

\begin{thebibliography}{99}
\scriptsize

\bibitem{des1977}
National Bureau of Standards (1977).
\textit{Data Encryption Standard (DES)}.
FIPS PUB 46.

\bibitem{bernstein2005}
Bernstein, D.J. (2005).
\textit{Cache-timing attacks on AES}.
\url{https://cr.yp.to/antiforgery/cachetiming-20050414.pdf}

\bibitem{sotocruz2024}
Soto-Cruz, J., et al. (2024).
\textit{Efficient Lightweight Cryptography for Power-Constrained MCUs}.
Technologies, 13(1), 3. MDPI.

\end{thebibliography}

\end{frame}

\end{document}
